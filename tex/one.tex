\documentclass{article}
% \documentclass [tikz,convert= {outfile=\jobname.svg}] {standalone}

\usepackage{amsmath}
\usepackage{tikz}

\begin{document}
\section{Gradient Descent}
\def\d{2.0}

\def\dX{6.0}
\def\dY{3.0}
\begin{center}
  \begin{tikzpicture}
    %%   (lable)  (x, y)     {\text}
    \node[shape=circle,draw=black, minimum size=1.5cm] (X1) at (-\dX,  0.50*\dY) {$x_1$};
    \node[shape=circle,draw=black, minimum size=1.5cm] (X2) at (-\dX, -0.50*\dY) {$x_2$};
    \node[shape=circle,draw=black, minimum size=1.5cm] (Y)  at (+\dX,         0) {$y$};

    \node[shape=circle,draw=black, minimum size=1.5cm] (N1) at (0,      \dY) {$\sigma,b_1^{(1)}$};
    \node[shape=circle,draw=black, minimum size=1.5cm] (N2) at (0, 0.33*\dY) {$\sigma,b_2^{(1)}$};
    \node[shape=circle,draw=black, minimum size=1.5cm] (N3) at (0,-0.33*\dY) {$\sigma,b_3^{(1)}$};
    \node[shape=circle,draw=black, minimum size=1.5cm] (N4) at (0,     -\dY) {$\sigma,b_4^{(1)}$};

    \path[-] (X1) edge node[above] {$w_{11}^{(1)}$} (N1);
    \path[-] (X1) edge                              (N2);
    \path[-] (X1) edge                              (N3);
    \path[-] (X1) edge                              (N4);
    %
    \path[-] (X2) edge                              (N1);
    \path[-] (X2) edge                              (N2);
    \path[-] (X2) edge                              (N3);
    \path[-] (X2) edge node[above] {$w_{24}^{(1)}$} (N4);

    \path[-] (N1) edge (Y);
    \path[-] (N2) edge (Y);
    \path[-] (N3) edge (Y);
    \path[-] (N4) edge (Y);

  \end{tikzpicture}
\end{center}

ooooo \\
%
% \begin{center}
%   \begin{tikzpicture}
%     \node (X) at (-\d, 0) {$x$};
%     \node[shape=circle,draw=black] (N1) at (0, 0) {$\sigma, b^{(1)}$};
%     \node[shape=circle,draw=black] (N2) at (\d, 0) {$\sigma, b^{(2)}$};
%     \node (Y) at ({2*\d}, 0) {$y$};
%     \path[->] (X) edge node[above] {$w^{(1)}$} (N1);
%     \path[->] (N1) edge node[above] {$w^{(2)}$} (N2);
%     \path[->] (N2) edge (Y);
%   \end{tikzpicture}
% \end{center}
If we keep decreasing the $\epsilon$ in our Finite Difference approach we effectively get the Derivative of the Cost Function.



Let's compute the derivatives of all our models. Throughout the entire paper $n$ means the amount of samples in the training set.

\subsection{Linear Model}


\begin{center}
  \begin{tikzpicture}
    \node (X) at ({-\d*0.75}, 0) {$x$};
    \node[shape=circle,draw=black] (N) at (0, 0) {$w$};
    \node (Y) at ({\d*0.75}, 0) {$y$};
    \path[->] (X) edge (N);
    \path[->] (N) edge (Y);
  \end{tikzpicture}
\end{center}

\begin{align}
  y &= x \cdot w
\end{align}

\subsubsection{Cost}

\begin{align}
  C(w) &= \frac{1}{n}\sum_{i=1}^{n}(x_iw - y_i)^2 \\
  C'(w)
       &= \left(\frac{1}{n}\sum_{i=1}^{n}(x_iw - y_i)^2\right)' = \\
       &= \frac{1}{n}\left(\sum_{i=1}^{n}(x_iw - y_i)^2\right)' \\
       &= \frac{1}{n}\sum_{i=1}^{n}\left((x_iw - y_i)^2\right)' \\
       &= \frac{1}{n}\sum_{i=1}^{n}2(x_iw - y_i)x_i
\end{align}

\subsection{One Neuron Model with 2 inputs}

\begin{center}
  \begin{tikzpicture}
    \node (X) at (-\d, 1) {$x$};
    \node (Y) at (-\d, -1) {$y$};
    \node[shape=circle,draw=black] (N) at (0, 0) {$\sigma, b$};
    \node (Z) at (\d, 0) {$z$};
    \path[->] (X) edge node[above] {$w_1$} (N);
    \path[->] (Y) edge node[above] {$w_2$} (N);
    \path[->] (N) edge (Z);
  \end{tikzpicture}
\end{center}
\begin{align}
  z &= \sigma(xw_1 + yw_2 + b) \\
  \sigma(x) &= \frac{1}{1 + e^{-x}} \\
  \sigma'(x) &= \sigma(x)(1 - \sigma(x))
\end{align}

\subsubsection{Cost}

\def\pd[#1]{\partial_{#1}}
\def\avgsum[#1,#2]{\frac{1}{#2}\sum_{#1=1}^{#2}}
\begin{align}
  a_i &= \sigma(x_iw_1 + y_iw_2 + b) \\
  \pd[w_1]a_i
      &= \pd[w_1](\sigma(x_iw_1 + y_iw_2 + b)) = \\
      &= a_i(1 - a_i)\pd[w_1](x_iw_1 + y_iw_2 + b) = \\
      &= a_i(1 - a_i)x_i \\
  \pd[w_2]a_i &= a_i(1 - a_i)y_i \\
  \pd[b]a_i &= a_i(1 - a_i) \\
  C &= \avgsum[i, n](a_i - z_i)^2 \\
  \pd[w_1] C
      &= \avgsum[i, n]\pd[w_1]\left((a_i - z_i)^2\right) = \\
      &= \avgsum[i, n]2(a_i - z_i)\pd[w_1]a_i = \\
      &= \avgsum[i, n]2(a_i - z_i)a_i(1 - a_i)x_i \\
  \pd[w_2] C &= \avgsum[i, n]2(a_i - z_i)a_i(1 - a_i)y_i \\
  \pd[b] C &= \avgsum[i, n]2(a_i - z_i)a_i(1 - a_i)
\end{align}

\subsection{Two Neurons Model with 1 input}

\begin{center}
  \begin{tikzpicture}
    \node (X) at (-\d, 0) {$x$};
    \node[shape=circle,draw=black] (N1) at (0, 0) {$\sigma, b^{(1)}$};
    \node[shape=circle,draw=black] (N2) at (\d, 0) {$\sigma, b^{(2)}$};
    \node (Y) at ({2*\d}, 0) {$y$};
    \path[->] (X) edge node[above] {$w^{(1)}$} (N1);
    \path[->] (N1) edge node[above] {$w^{(2)}$} (N2);
    \path[->] (N2) edge (Y);
  \end{tikzpicture}
\end{center}

\begin{align}
  a^{(1)} &= \sigma(xw^{(1)} + b^{(1)}) \\
  y &= \sigma(a^{(1)}w^{(2)} + b^{(2)})
\end{align}

The superscript in parenthesis denotes the current layer. For example $a_i^{(l)}$ denotes the activation from the $l$-th layer on $i$-th sample.

\subsubsection{Feed-Forward}

\begin{align}
  a_i^{(1)} &= \sigma(x_iw^{(1)} + b^{(1)}) \\
  \pd[w^{(1)}]a_i^{(1)} &= a_i^{(1)}(1 - a_i^{(1)})x_i \\
  \pd[b^{1}]a_i^{(1)} &= a_i^{(1)}(1 - a_i^{(1)}) \\
  a_i^{(2)} &= \sigma(a_i^{(1)}w^{(2)} + b^{(2)}) \\
  \pd[w^{(2)}]a_i^{(2)} &= a_i^{(2)}(1 - a_i^{(2)})a_i^{(1)} \\
  \pd[b^{(2)}]a_i^{(2)} &= a_i^{(2)}(1 - a_i^{(2)}) \\
  \pd[a_i^{(1)}]a_i^{(2)} &= a_i^{(2)}(1 - a_i^{(2)})w^{(2)}
\end{align}

\subsubsection{Back-Propagation}

\begin{align}
  C^{(2)} &= \avgsum[i, n] (a_i^{(2)} - y_i)^2 \\
  \pd[w^{(2)}] C^{(2)}
            &= \avgsum[i, n] \pd[w^{(2)}]((a_i^{(2)} - y_i)^2) = \\
            &= \avgsum[i, n] 2(a_i^{(2)} - y_i)\pd[w^{(2)}]a_i^{(2)} = \\
            &= \avgsum[i, n] 2(a_i^{(2)} - y_i)a_i^{(2)}(1 - a_i^{(2)})a_i^{(1)} \\
  \pd[b^{(2)}] C^{(2)} &= \avgsum[i, n] 2(a_i^{(2)} - y_i)a_i^{(2)}(1 - a_i^{(2)}) \\
  \pd[a_i^{(1)}]C^{(2)} &= \avgsum[i, n] 2(a_i^{(2)} - y_i)a_i^{(2)}(1 - a_i^{(2)})w^{(2)} \\
  e_i &= a_i^{(1)} - \pd[a_i^{(1)}]C^{(2)} \\
  C^{(1)} &= \avgsum[i, n] (a_i^{(1)} - e_i)^2 \\
  \pd[w^{(1)}]C^{(1)}
            &= \pd[w^{(1)}]\left(\avgsum[i, n] (a_i^{(1)} - e_i)^2\right) =\\
            &= \avgsum[i, n] \pd[w^{(1)}]\left((a_i^{(1)} - e_i)^2\right) =\\
            &= \avgsum[i, n] 2(a_i^{(1)} - e_i)\pd[w^{(1)}]a_i^{(1)} =\\
            &= \avgsum[i, n] 2(\pd[a_i^{(1)}]C^{(2)})a_i^{(1)}(1 - a_i^{(1)})x_i \\
  \pd[b^{1}]C^{(1)} &= \avgsum[i, n] 2(\pd[a_i^{(1)}]C^{(2)})a_i^{(1)}(1 - a_i^{(1)})
\end{align}

\subsection{Arbitrary Neurons Model with 1 input}

Let's assume that we have $m$ layers.

\subsubsection{Feed-Forward}

Let's assume that $a_i^{(0)}$ is $x_i$.

\begin{align}
  a_i^{(l)} &= \sigma(a_i^{(l-1)}w^{(l)} + b^{(l)}) \\
  \pd[w^{(l)}]a_i^{(l)} &= a_i^{(l)}(1 - a_i^{(l)})a_i^{(l-1)} \\
  \pd[b^{(l)}]a_i^{(l)} &= a_i^{(l)}(1 - a_i^{(l)}) \\
  \pd[a_i^{(l-1)}]a_i^{(l)} &= a_i^{(l)}(1 - a_i^{(l)})w^{(l)}
\end{align}

\subsubsection{Back-Propagation}

Let's denote $a_i^{(m)} - y_i$ as $\pd[a_i^{(m)}]C^{(m+1)}$.

\begin{align}
  C^{(l)} &= \avgsum[i, n] (\pd[a_i^{(l)}]C^{(l+1)})^2 \\
  \pd[w^{(l)}]C^{(l)} &= \avgsum[i, n] 2(\pd[a_i^{(l)}]C^{(l+1)})a_i^{(l)}(1 - a_i^{(l)})a_i^{(l-1)} =\\
  \pd[b^{(l)}]C^{(l)} &= \avgsum[i, n] 2(\pd[a_i^{(l)}]C^{(l+1)})a_i^{(l)}(1 - a_i^{(l)}) \\
  \pd[a_i^{(l-1)}]C^{(l)} &= \avgsum[i, n] 2(\pd[a_i^{(l)}]C^{(l+1)})a_i^{(l)}(1 - a_i^{(l)})w^{(l)}
\end{align}

\end{document}
